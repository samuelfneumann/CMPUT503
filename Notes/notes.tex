\documentclass[13pt]{article}

\usepackage{amsmath}
\usepackage{bm}

\title{CMPUT503: Experimental Mobile Robotics}
\author{Samuel Neumann}

\begin{document}
\maketitle

\section{Introduction}
\hfill  \textbf{Thu 05 Jan 2023}

\hfill \\

\noindent
\textbf{Goals}: appreciate robotics, and understand what is possible and how, get all the basic knowledge you need in
this class to jump into a robotics job (i.e. pass interviews)

\hfill \\
\noindent
Math is not the focus of this course.

\hfill \\
\noindent
If you do the work, you get the grade. But, you probably will have to come into the lab after lab hours (i.e. more than
3 in-lab hours)

\hfill \\
\noindent
Cut corners wherever you can, use already-built packages for assignments/labs

\hfill \\
\noindent
Lowest two lab grades will get dropped!

\hfill \\
\noindent
Look into DuckieTown before the first lab.

\hfill \\

\section{Robot Architecture and Locomotion}%
\hfill \textbf{Tue 10 Jan 2023} \\

\noindent
\textit{Architecture} can be thought of as the interaction of hardware and software.

\subsection{Architectures}%
\noindent
\begin{itemize}
	\item Reactive Architecture
	\begin{itemize}
		\item \textit{Actions} are directly triggered by \textit{sensors}.
		\item No representations of the environment
		\item Predefined, fixed response to situation
		\item \textit{Fast} response to changes in the environment
		\item Limitations: no memory (unable to count), low computing power, no machine learning, knowledge of the world
			is limited by the range of its sensors, cannot \textit{undo} and incorrect action, not possible to plan
			ahead, unable to recover from actions which fail silently, etc.
	\end{itemize}
	\item Deliberative Architecture
	\begin{itemize}
		\item Organized by decomposing the required system functionality into \textit{concurrent modules} or components
		\item Problems: overall complexity of the system may grow, hard to offer real-time guarantees on performance
		\begin{itemize}
			\item Solving any given problem takes longer than an equivalent \textit{reactive} implementation
			\item Solving different problems take different amount of time
		\end{itemize}
	\end{itemize}
	\item Subsumption Architecture
	\item ROS?
\end{itemize}

\noindent
\textit{Temporal Decomposition} distinguishes between processes that have varying real-time and non-real-time demands

\noindent
\textbf{If you fail to meet a hard real-time requirement, then your system is broken.} E.g.: driving off a cliff.

\textbf{Rodney A. Brooks}: tried to make a realistic robot with the abilities of an insect. He developed the \textit{subsumption
architecture}. This architecture puts different importance/prioritization on different tasks. E.g. I really don't want
to fall off the cliff, but if I'm not falling off a cliff, then don't bump into any objects, etc.

\subsection{What is ROS?}%
\noindent
\begin{itemize}
	\item A \textit{meta} operating system for robots
	\item An \textbf{architecture} for distribution inter-process/inter-machine communication and configuration
	\item ROS isa \textit{peer-to-peer} robot middleware package
	\item In ROS, all major functionality is broken into nodes that communicate with each other using messages sent
		on topics
	\item \textbf{Not} a programming language
	\item \textbf{Not} a \textit{hard real-time} architecture
	\item ...
\end{itemize}

\subsubsection{ROS Nodes}%
\noindent
\begin{itemize}
	\item \textbf{Node}: a process that performs some computation
	\item Typically we try to divide the entire software into different modules -- eac hon ei srun onver a single or
		multiple nodes.
	\item  Nodes are combined together into graph and communicated with each other using streaming topics. Nodes
		\textbf{do not} communicate with each other directly, otherwise we get into the craziness and complicationes of
		the other architectures.
\end{itemize}

\subsubsection{ROS Topics}%
\noindent
\begin{itemize}
	\item \textbf{Topic}: named buses over which nodes exchange messages to ROS master
	\item Topics have \textbf{anonymous} publish/subscribe semantics
	\item Can be \textit{multiple} publisher and subscribers to a topic
	\item Strongly typed
\end{itemize}

\subsubsection{ROS Messages}%
\noindent
\begin{itemize}
	\item \textbf{Message}: is a simple data structure, comprising typed fields.
	\item Can contain headers with metadata
\end{itemize}

\section{Locomotion}%
\noindent
\textbf{Locomotion}: enabling robots to move

\hfill

\noindent
Wheel locomotion is simple, safer, and stable:

\hfill

\noindent
\textbf{Key issues for wheeled locomotion}:
\begin{itemize}
	\item Stability
	\begin{itemize}
		\item Number and geometry of contact points
		\item Centre of gravity
		\item Static/dynamic stability
		\item Inclination of terrain
	\end{itemize}
	\item Characteristics of contact
	\begin{itemize}
		\item Contact point/path size and shape
		\item Angle of contact
		\item Friction
	\end{itemize}
	\item Type of environment
\end{itemize}

\hfill

\noindent
\textbf{Legged mobile robots} can adapt to human terrain, but they have lots of limitations: power and mechanical complexity,
high degrees of freedom, control system complexity, etc.

\subsection{Leg Locomotion}%
\noindent
\textbf{Degree of Freedom} (DOF): joints or axes in motion

\hfill

\noindent
A minimum of two DOFs is required to move a leg forward, but usually legs have three degrees of freedom. A fourth DOF is
needed for the angle joint, which might improve walking.

\hfill

\noindent
As you add more DOFs, the complexity of the system increases very fast.

\hfill

\noindent
Arms need 6 or 7 DOFs.

\hfill

\noindent
\textit{Often, clever mechanical design can perform the same operations as complex active control circuitry.}

\hfill \\ \hfill \\

\hfill \textbf{Thu 12 Jan 2023}

\noindent
\textbf{Gait control}: how legs are coordinated for movement, e.g. crawl, trot, pace, bound, pronk, gallop, etc.

\hfill \\
\noindent
Number of gaits is $(2k - 1)!$ where $k$ is the number of legs.

\hfill \\
\noindent
\textbf{Cost of transportation}:
\begin{itemize}
	\item How much energy a robot uses to travel a certain distance
	\item Usually normalized by the robot weight
	\item Measured in $\frac{J}{N-m}$
	\item Driving on wheels has very low cost of transportation.
\end{itemize}

\hfill \\
\noindent
Legged robot control should be designed to better exploit the dynamics of the system. For example passive dynamic
walking.

\subsection{Wheeled Mobile Robots}%
\noindent
Wheels are \textit{the most popular locomotion mechanisms}:
\begin{itemize}
	\item Highly efficient
	\item Simple mechanical implementation
	\item Balancing is not \textit{usually} a problem, but a suspension system is needed to allow all wheel to maintain
		ground contact on uneven terrain.
\end{itemize}

\hfill \\
\noindent
We will focus on:
\begin{itemize}
	\item Traction
	\item Stability
	\item Maneuverability
	\item Control -- we will mostly focus on this
\end{itemize}

\hfill \\
\noindent
Wheel Designs:
\begin{itemize}
	\item Standard wheels
	\begin{itemize}
		\item 2 DOFs
	\end{itemize}
	\item Castor wheels
	\begin{itemize}
		\item 2 DOFs
		\item Can move while the contact point of the wheel stays the same.
		\item If something is super heavy, it's easier to get momentum with these wheels
		\item Harder to control
	\end{itemize}
	\item Swedish (Omni) wheels
	\begin{itemize}
		\item 3 DOFs
	\end{itemize}
	\item Ball or spherical wheel
	\begin{itemize}
		\item 3 DOFs
		\item Balled computer mice used these
		\item Suspension issue -- hard to get suspension on these
	\end{itemize}
\end{itemize}

\hfill \\
\noindent
Stability of a vehicle is guaranteed with 3 wheels -- centre of gravity is required to be within the triangle formed by
ground contact points of wheels.

\hfill\\
\noindent
Stability is improved by 4 and more wheels

\hfill \\
\noindent
\textbf{Holonomic}: can move in any direction

\hfill \\
\noindent
Combining \textbf{actuation} and \textbf{steering} on one wheel makes design complex and adds additional errors for
odometry.

\hfill \\
\noindent
Static stability with two wheels can be achieved by \textbf{ensuring centre of mass is below the wheel axis} or by using
fancy controllers.

\subsection{Motion Control}%
\begin{itemize}
	\item Kinematic/dynamic model of the robot comes in here
	\item Model the interaction between the wheel and ground
	\item Definition of required motion -- which motion is required:
	\begin{itemize}
		\item Speed control
		\item Position control
	\end{itemize}
	\item Control law that satisfies the requirements
\end{itemize}

\hfill \\
\noindent
\textbf{Kinematics}: Description of mechanical behaviour of the robot for design and control. E.g. hwo a robot will move
given motor inputs.

\hfill \\
\noindent
\textbf{Mobile robots} can move unbounded with respect to their enviornment:
\begin{itemize}
	\item No direct way to measure robot's position
	\item Position must be integrated over time
	\item Leads to inaccuracies of the position (motion) estimate
\end{itemize}

\hfill \\
\noindent
Some important definitions:
\begin{itemize}
	\item \textbf{Configuration}: complete specification of the position of every point of the stytem. Position and
		orientation. Also called a \textbf{pose}.
	\begin{itemize}
		\item For use: $(x, y, z)$
	\end{itemize}
	\item \textbf{Configuration space}: space of all possible configurations (can be thought of as one of the robot's
		frame of references, i.e. we can change the workspace and consider the robot as e.g. a point)
	\item \textbf{Workspace}: the 2D or 3D \textbf{ambient} space the robot is in
\end{itemize}

\section{Kinematics}%
\noindent
Robot knows how movement relative to centre of rotation (P). This is \textbf{not} the same as knowing how to robot moves
in the real world.

\hfill \\
\noindent
Two frames we need to worry about:
\begin{itemize}
	\item Initial frame: the world
	\item Robot frame: think of yourself as the robot
\end{itemize}

\hfill \\
\noindent
Robot is at initial frame $\xi_{i} = (x_{i}, y_{i}, \theta_{i})$. We want to get to some location, but we can't control $(x_i,
y_{i}, \theta_{i})$ directly.

\hfill \\
\noindent
The robot can know the speed of wheel $i$: $\dot \phi_{i}$, steering angle of steerable wheel $i$: $\beta_{i}$, and
speed with which steering angle for wheel $i$ is changing: $\dot \beta_{i}$.

\hfill \\
\noindent
In the robot frame, we have $\xi_{r} = (x_{r}, y_{r}, \theta_{r})$. Since in the robot frame $\theta_{r} = 0$, we simply
define $\theta_{r} = \theta_{i}$. We have: \[
	f(\dot \phi_{i}, \dot \phi_{r}) = [\dot x_{i} \quad \dot y_{i} \quad \dot \theta_{i} ]^{\top}
.\]

\hfill \\
\noindent
What we want to do now is:
\begin{equation}
	\begin{pmatrix} \dot x_{r} \\ \dot y_{r} \\ \dot \theta_{r} \end{pmatrix} =
	\bm{R}_{\theta} \begin{pmatrix} \dot x_{i} \\ \dot y_{i} \\ \dot \theta_{i} \end{pmatrix} \qquad
	\text{where }
	\bm{R}_{\theta} = \begin{pmatrix} \cos(\theta) & \sin(\theta) & 0 \\ - \sin(\theta) & \cos(\theta) & 0 \\ 0 & 0 & 1 \end{pmatrix}
\end{equation}
where $\theta = \theta_{i} = \theta_{r}$

\hfill \\
\noindent
Still, this isn't what we want, we want the reverse of the (inverse-kinematics):
\begin{equation}
	\begin{pmatrix} \dot x_{i} \\ \dot y_{i} \\ \dot \theta_{i} \end{pmatrix} =
	\bm{R}_{\theta}^{-1} \begin{pmatrix} \dot x_{r} \\ \dot y_{r} \\ \dot \theta_{r} \end{pmatrix} \qquad
	\text{where }
	\bm{R}_{\theta}^{-1} = \begin{pmatrix} \cos(\theta) & \sin(\theta) & 0 \\ - \sin(\theta) & \cos(\theta) & 0 \\ 0 & 0 & 1 \end{pmatrix}
\end{equation}
this tells us how to change the robot's wheels to get somewhere in the world. In our case, $\dot y_{r} = 0$ always.

\hfill \\
\noindent
Constraints and assumptions:
\begin{itemize}
	\item Movement on a horizontal plane
	\item Point contact of wheels
	\item Wheels are not deformable
	\item Pure rolling: velocity is 0 at contact point
	\item ...
\end{itemize}

\hfill \\
\noindent
Differential Drive:
\begin{itemize}
	\item The \textbf{differential drive} is a two-wheeled drive system with independent actuators for each wheel. The name
		refers to the fact that the motion vector of the robot is the sum of the independent wheel motions, and so
		turning can be accomplished by rotating the wheels at different speeds. The drive
		wheels are usually placed on each side of the robot and toward the front.
	\item Wheels rotate at $\dot \phi$
	\item Each wheel contributes $\frac{r \dot \phi}{2}$ to the motion of centre of rotation.
	\item Speed: sum of two wheels
	\item Rotation: due to the right wheel is $\omega_{r} = \frac{r \dot \phi}{2 l}$ counterclockwise about left wheel,
		where $l$ is the distance between the wheel and centre of rotation.
	\item Combining components
	\begin{equation}
		\begin{pmatrix} \dot x_{r} \\ \dot y_{r} \\ \dot \theta_{r} \end{pmatrix}  = \begin{pmatrix} \frac{r \dot
	\phi_{r}}{2} + \frac{r \dot \phi_{l}}{2} \\ 0 \\ \ldots \frac{r\dot \phi_{r}}{2l} - \text{what is this? get from notes} \end{pmatrix}
	\end{equation}
\end{itemize}
now, what is the change in the initial frame of reference?


\hfill

\hfill \textbf{Tue 17 Jan 2023} \\

\noindent
Kinematics works in a perfect world, with all our above assumptions satisfied, and where the robot is a point mass. In
this case, opened-loop controllers like kinematics work.

\hfill \\

\noindent
Sliding constraint:
\begin{itemize}
	\item Standard wheel has no lateral motion, we can have steered standard wheels or steered caster wheels
	\item Move in circle whose centre is on \textit{zero motion line} through the axis
\end{itemize}

\hfill

\noindent
\textbf{Degree of Mobility}: number of degrees of freedom of robot chassis that can be immediately manipulated through
changes in wheel velocity.

\hfill

\noindent
\textbf{Degree of Maneuverability}: the overall degrees of freedom that a robot can manipulate: $\delta_{M} = \delta_{m}
+ \delta_{s}$

\hfill

\noindent
We may want a robot to be \textbf{redundant} if it needs to get to a certain point. If one path is blocked, it can take
another path e.g. Think of a manipulator attempting to get its end effector to a certain position. If one trajectory is
blocked, it can make another.

\section{Manipulators}%

A \textbf{manipulator} is some robot that manipulates (physically alters) something in the real world, but not its own
position, at least as a primary goal. Desirable in dangerous, dirty, or dull workspaces.

\hfill

\noindent



\end{document}
